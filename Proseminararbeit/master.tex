\documentclass{article}
\usepackage[utf8]{inputenc}
\usepackage[T1]{fontenc}
\usepackage[german]{babel} 
\usepackage{amsmath,amssymb,amsthm}
\usepackage{makeidx}
\usepackage{graphicx}
\usepackage{stmaryrd}
%\usepackage{hyperref}
\usepackage{xspace}
\usepackage{url}
\usepackage{comment}
\newcommand{\say}[1]{\marginpar{#1}}
\usepackage{multirow}
\usepackage{rotating}
\usepackage{colortbl}
\usepackage{subfigure}
\usepackage{appendix}
\usepackage{sidecap}
\usepackage{capt-of}
\usepackage{savefnmark}
 \usepackage{longtable}
\usepackage[pdftex]{lscape}
\usepackage{afterpage}
\usepackage{tikz}
\usepackage[pdfborder={0 0 0}]{hyperref}
\usepackage{dsfont}
%\usepackage{enumitem} 
\usepackage{morefloats}
\usepackage{setspace}
%\usepackage{slashbox}
\usepackage[oldenum,olditem]{paralist}
\usepackage[final]{pdfpages}
%\usepackage{algorithm}
%\usepackage{algpseudocode}
\usepackage{here}
\usepackage{amsmath}
\usepackage[linesnumbered,ruled]{algorithm2e}
%\usepackage{algorithm}
%\usepackage{algpseudocode}
%\usepackage{pifont}
\usepackage{textgreek}

%tikz graphics
\usepackage{pgf}
\usepackage{graphics}
\usepackage[margin=1in]{geometry}
\usetikzlibrary{arrows,automata}
 %to enable backgrounds in tikz graphics -> edges do not overlay vertices
\usetikzlibrary{backgrounds}
\pgfdeclarelayer{myback}
\pgfsetlayers{background,myback,main}

\usetikzlibrary{calc}
\usetikzlibrary{decorations.pathmorphing,patterns}
\usetikzlibrary{calc,patterns,decorations.markings}
\usetikzlibrary{decorations.pathreplacing}
\newcommand{\argmin}{\operatornamewithlimits{argmin}}
\newcommand{\argmax}{\operatornamewithlimits{argmax}}

\newcommand{\vars}[1]{\textsc{#1}}
\newcommand{\meth}[1]{\texttt{\textbf{#1}}}
\newcommand{\true}{\textbf{true}}
\newcommand{\false}{\textbf{false}}
\usepackage[textsize=tiny]{todonotes}
\usepackage{enumerate} 
\usepackage{memhfixc}
\usepackage{hyphenat} 

\makeatletter
\let\NAT@parse\undefined
\makeatother

\usepackage[sort,numbers]{natbib}
\usepackage{afterpage}
\usepackage{refcount}
\usepackage{graphicx}

%pagelayout
\oddsidemargin -12pt
\evensidemargin -12pt
\topmargin -32pt
\textheight 650pt
\textwidth 460pt
\def\baselinestretch{1.1}

\newtheoremstyle{dotless}{}{}{\itshape}{}{\bfseries}{}{ }{}
\theoremstyle{dotless}

% Theorem definitions
\newtheorem{definition}{Definition}[section]
\newtheorem{theorem}{Theorem}[section]
%\newtheorem{thm}{Theorem}[chapter]
\newtheorem{corollary}{Korollar}[section]
%\newtheorem{cor}[theorem]{Corollary}
\newtheorem{lemma}{Lemma}[section]
\newtheorem{example}{Beispiel}[section]
\newtheorem{note}{Anmerkung}[section]
%\newtheorem{lem}[theorem]{Lemma}
%\newtheorem{fact}[theorem]{Fact}
%\newtheorem{proposition}[theorem]{Proposition} \newtheorem{rem}{Remark}
%\newtheorem{obs}[theorem]{Observation}
%\newtheorem{exam}[theorem]{Example}
%\newenvironment{example}{%
%\italicenvfalser
%\begin{exam}}{\end{exam}\italicenvtrue}
%\newcommand{\keywords}[1]{\par\noindent
%{\small{\em Keywords\/}: #1}}
%\newcommand{\abstract}[1]{\par\noindent
%{\small{\bf Abstract\/}: #1}}
%\newcommand{\authors}[1]{\par\noindent
%{\small{\bf Authors\/}: #1}}
%\newcommand{\savefootnote}[2]{\footnote{\label{#1}#2}}
%\newcommand{\repeatfootnote}[1]{\textsuperscript{\ref{#1}}}

\makeindex

\begin{document}


%
% Vorspann
%


\begin{titlepage}
\vspace*{1cm}
  \begin{center}   
	    \LARGE \textbf{Anna Jellen} \\
      \LARGE \textbf{Kerstin Maier}
  \end{center}
\vspace{1cm}
  \begin{center}
 \Huge \textbf{Der Nullstellensatz in der Diskreten Mathematik} 
  \end{center}
\vspace{1cm}
\begin{center}
\huge  \textsc{Proseminararbeit}
\end{center}
\vspace{1cm}
\begin{center}
\LARGE Alpen-Adria-Universit\"{a}t Klagenfurt\\
\LARGE Fakult\"{a}t f\"{u}r  Technische Wissenschaften
\end{center}
\vspace{1.5cm}
\Large
\begin{center}
 Betreuer:  \= Assoc.Prof. Dipl.-Ing.Dr. Angelika Wiegele 
 \\ Institut f\"{u}r Mathematik, Universität Klagenfurt\\ \end{center}
  \begin{flushright} Sommersemester 2016 \end{flushright}
\end{titlepage}
	

\addcontentsline{toc}{section}{Abstract}
\vspace*{0.2cm}\section*{Abstract}
In dieser Proseminararbeit widmen wir uns dem Hilbert'schen Nullstellensatz, der zu den klassischen Theoremen in der Algebraischen Geometrie zählt. Im Fokus steht die Verwendung des Nullstellensatzes zum Beweisen der Unlösbarkeit von kombinatorischen Optimierungsproblemen, wie zum Beispiel der 3-Färbbarkeit eines Graphen. \\

\noindent Zu Beginn werden wir auf Grundbegriffe aus der Graphentheorie eingehen. Danach wird der Nullstellensatz und seine Anwendungen in der Diskreten Optimierung genauer behandelt. Zum Schluss widmen wir uns noch theoretischen Resultaten und dem Beweis des Nullstellensatzes. 

%%% Local Variables: 
%%% mode: latex
%%% TeX-master: "master"
%%% End: 

\nocite{*}

\renewcommand{\contentsname}{Inhaltsverzeichnis}

\tableofcontents




\part*{Nullstellensatz}


\section{Einleitung}

Zu Beginn möchten wir auf einige Grundbegriffe in der Graphentheorie und der (Linearen) Algebra eingehen. Dieses Kapitel soll als Nachschlagewerk für die restlichen Kapitel dieser Arbeit dienen. Hierbei werden auch die Begriffe Stable Set, k-Färbbarkeit und Maximaler Schnitt eingeführt, das Finden solcher graphentheoretischer Probleme fällt in die Klasse der NP-vollständigen Probleme. Des Weiteren werden wir in \ref{Grundbegriffe} diese drei Probleme in Polynomialdarstellung  bringen, was dazu führt, dass wir den Hilbertschen Nullstellensatz darauf anwenden können. Wir werden in dieser Arbeit auf dieses zentrale Resultat der Algebraischen Geometrie eingehen und zeigen, dass sich daraus ein Algorithmus ableiten lässt, der entscheidet, ob die angegebenen Probleme lösbar oder unlösbar sind. 

\section{Grundbegriffe}  \label{Grundbegriffe}

\subsection{Grundbegriffe aus der (Linearen) Algebra}

\begin{definition}
Es sei H eine Menge mit einer inneren Verknüpfung $\cdot : H \times H \rightarrow H$. Es heißt $(H,\cdot)$ eine \textcolor{blue}{Halbgruppe}, wenn $\forall a,b,c \in H$ gilt \cite{Karpfinger}: 
\begin{align*}
(a \cdot b) \cdot c = a \cdot ( b \cdot c ).
\end{align*}
\end{definition}

\begin{definition}
Es sei G eine nichtleere Menge mit einer inneren Verknüpfung $\cdot : G \times G \rightarrow G$. Es heißt $(G,\cdot)$ eine \textcolor{blue}{Gruppe}, wenn $\forall a,b,c \in G$ gilt \cite{Karpfinger}: 
\begin{enumerate}
\item $(a \cdot b) \cdot c = a \cdot ( b \cdot c ).$
\item $\exists e \in G$ mit $ e \cdot a = a = a \cdot e. $
\item $\forall a \in G :\,\exists a' \in G $ mit $ a' \cdot a = e = a \cdot a'.$
\end{enumerate}
\end{definition}


\begin{definition}
Es sei R eine Menge mit den inneren Verknüpfung $+ : R \times R \rightarrow R$ und $\cdot : R \times R \rightarrow R$. Es heißt $(R,+,\cdot)$ ein \textcolor{blue}{Ring}, wenn $\forall a,b,c \in R$ gilt \cite{Karpfinger}: 
\begin{enumerate}
\item $a+b=b+a.$
\item $a+(b+c)=(a+b)+c$.
\item $\exists 0 $ (Nullelement) in $ R : 0+a= a \quad \forall a \in R$.
\item $\forall a \in R :\,\exists -a \in R$ (inverses Element) $: a + (-a) = 0$.
\item $(a \cdot b) \cdot c = a \cdot ( b \cdot c ).$
\item $a(b+c)=ab+ac$ und $(a+b)c = ac+bc$.
\end{enumerate}
\end{definition}

\begin{definition}
Es sei K eine Menge mit den inneren Verknüpfung $+ : K \times K \rightarrow K$ und $\cdot : K \times K \rightarrow K$. Es heißt $(K,+,\cdot)$ ein \textcolor{blue}{Körper}, wenn $\forall a,b,c \in K$ gilt \cite{Karpfinger}: 
\begin{enumerate}
\item $a+b=b+a.$
\item $a+(b+c)=(a+b)+c$.
\item $\exists 0 $ (Nullelement) in $ K : 0+a= a \quad \forall a \in K$.
\item $\forall a \in K :\,\exists -a \in K$ (inverses Element) $: a + (-a) = 0$.
\item $ab=ba$.
\item $(a \cdot b) \cdot c = a \cdot ( b \cdot c ).$
\item $\exists 1 \neq 0$ (Einselement) in $ K: 1a=a \quad \forall a \in K$.
\item $\forall a \in K \textbackslash \{ 0 \} :\quad \exists a^{-1}\in K $ (inverses Element) $: aa^{-1}=1.$
\item $a(b+c)=ab+ac$ und $(a+b)c = ac+bc$.
\end{enumerate}
\end{definition}

\begin{definition}
Es sei 
\begin{align*}
R[X] = \{\sum_{i\in\mathbb{N}_0} a_i X^i |a_i \in R \text{ und} a_i = 0 \text{ für fast alle i} \,\in \mathbb{N}_0 \}
\end{align*}
ein kommutativer Ring mit Einselement 1. \\
Man nennt R[X] den \textcolor{blue}{Polynomring} in der Unbestimmten X über R. \cite{Karpfinger}
\end{definition}

\begin{definition}
Man nennt $K \subseteq E$ einen \textcolor{blue}{Teilkörper} von E und E einen \textcolor{blue}{Erweiterungskörper} von K sowie E/K eine \textcolor{blue}{Körpererweiterung}, wenn K ein Teilring von E und als solcher ein Körper ist, d.h. wenn gilt\cite{Karpfinger}:
\begin{align*}
a, b, c\in K, \quad c \neq 0 \quad \Rightarrow \quad a-b,ab,c^{-1}\in K. 
\end{align*} 
\end{definition}

\begin{definition}
Es sei L/K eine Körpererweiterung. Ein Element $a \in L$ heißt:
\begin{itemize}
\item \textcolor{blue}{algebraisch über} K, wenn es ein von 0 verschiedenes Polynom $P \in K[X]$ gibt mit $P(a) = 0$.
\item \textcolor{blue}{transzendent über} K, wenn es nicht algebraisch ist, d.h. für $P \in K[X]$ gilt $P(a) = 0$ nur für das Nullpolynom $P = 0$. \cite{Karpfinger}  
\end{itemize} 
\end{definition}

\begin{definition}
Der Körper K heißt \textcolor{blue}{algebraisch abgeschlossen}, wenn jedes nicht konstante Polynom aus K[X] eine Nullstelle in K hat. \cite{Karpfinger}
\end{definition}

\begin{definition}
Ein Erweiterungskörper von K wird \textcolor{blue}{algebraischer Abschluss} von K genannt, wenn er algebraisch über K und algebraisch abgeschlossen ist. \cite{Karpfinger}
\end{definition}

\begin{definition}
Sei F = $\{f_1,\ldots,f_m\} \subseteq \mathbb{K}[x_1,\ldots,x_n]$ eine Menge von Polynomen. Ein Vector $\overline{x} \in \overline{\mathbb{K}^n}$ ist eine \textcolor{blue}{Lösung des Systems} $f_1 = f_2 = \ldots = f_k=0$ ($F = 0$), wenn $f_i(\overline{x})=0 \quad \forall i=1,\dots,k$.\\
Die \textcolor{blue}{Varietät} $V(f_1,f_2,\ldots,f_k)$ ($V(F)$) ist die Menge aller Lösungen von $F=0$ in $\overline{\mathbb{K}^n}$.\cite{Ausgangsartikel}
\end{definition}

\begin{definition}
Die Menge von Polynomen beschreibt ein Ideal $I \subseteq \mathbb{K}[x_1,\ldots,x_n]$, wenn die folgenden Eigenschaften erfüllt sind \cite{Ausgangsartikel}:
\begin{enumerate}
\item $0 \in I$
\item $f,g \in I \Rightarrow f+g \in I$
\item $f \in I, g \in \mathbb{K}[x_1,\ldots,x_n] \Rightarrow fg \in I$
\end{enumerate}
Sei $F = \{f_1,\ldots,f_m\} \subseteq \mathbb{K}[x_1,\ldots,x_n]$, dann ist
\begin{align*}
\left\langle F \right\rangle = \left\{f = \sum_{i=1}^k h_if_i: \quad f_i \in F, h_i \in \mathbb{K}[x_1,\ldots,x_n] \quad \forall i = 1,\ldots,k\right\} \subseteq \mathbb{K}[x_1,\ldots,x_n]
\end{align*}
das \textcolor{blue}{polynomielle Ideal} erzeugt von F in $\mathbb{K}[x_1,\ldots,x_n]$.
\end{definition}
%\begin{definition}
%Für einen endlichen Körper $\Bbb F_p$ der Primzahl-Ordnung $p$ ist der algebraische Abschluss ein abzählbarer unendlicher Körper %der Charakteristik $p$ und enthält für jede natürliche Zahl n einen Teilkörper der Ordnung $p^n$, er besteht sogar aus der Vereinigung dieser Teilkörper. \cite{Karpfinger}
%\end{definition}

%\begin{definition}
%Charakteristik 
%\end{definition}

\subsection{Grundbegriffe der Graphentheorie} 

\begin{definition}
Ein \textcolor{blue}{Graph} G ist ein Paar G = (V,E) disjunkter Mengen mit $E \subseteq V\times V$. Die Elemente von $V$ nennt man Knoten und die Elemente von $E$ Kanten. \cite{Diestel}
\end{definition}

\begin{definition}
Zwei Knoten $x,y \in V(G)$ sind \textcolor{blue}{adjazent} in G und heißen \textcolor{blue}{Nachbarn} von einander, wenn $xy \in E(G)$. \cite{Diestel}
\end{definition}

\subsubsection*{Stable Set}

\begin{definition}
Paarweise nicht benachbarte Knoten oder Kanten nennt man \textcolor{blue}{unabhängig}. Eine Teilmenge von V oder E heißt unabhängig, wenn ihre Elemente paarweise nicht benachbart sind. Unabhängige Knotenmengen nennt man auch \textcolor{blue}{stabil}. \cite{Diestel}
\end{definition}

\begin{definition}
Ein \textcolor{blue}{Stable Set} ist eine unabhängige Knotenmenge. $\alpha (G)$ ist die maximale Größe eines Stable Sets vom Graph $G$.
\end{definition}


\begin{figure}[h]
\begin{center}
\resizebox{.5\linewidth}{!}{
\begin{tikzpicture}[->,>=stealth',shorten >=1pt,auto,node distance=2.5cm,
                    semithick]
  \tikzstyle{every state}=[draw=black,text=black]
	\tikzstyle{every edge} = [draw,thick,-]

  \node[state] 	(0)              	{1};
  \node[state,fill=red]  (1) [above right of=0] 	{2};
  \node[state,fill=red] 	(2) [right of=0]  {3};
  \node[state]  (3) [below of=0] 	{4};
  \node[state,fill=red] 	(4) [right of=3]  {5};
  \node[state]  (5) [right of=2] 	{6};
 

  
  \path[draw=black,thick] 
  		%% drone 1
  		(0)  edge node {} (2)
  		(3)  edge  node {} (4)
  		(5)  edge  node {} (4)
  		(1)  edge  node {} (0)
  		(2)  edge  node {} (3)
  		(1)  edge  node {} (5)
			(2)  edge  node {} (5)
			(3)  edge  node {} (5)
  		;
\end{tikzpicture}
}
\caption[Stable Set]{Stable Set}
  \label{StableSet}
\end{center}
\end{figure}

\subsubsection*{k-Färbbarkeit}

\begin{definition}
Die \textcolor{blue}{k-Färbung} eines Graphen ist eine Abbildung $f: V(G) \rightarrow S$ mit $xy \in E(G): f(x) \neq f(y)$, wobei S Menge der Farben und $\vert S \vert = k$. \cite{Diestel, West}
\end{definition}

\todo[inline]{chromatische Zahl}

\begin{figure}[htp]
\begin{center}
\resizebox{.5\linewidth}{!}{
\begin{tikzpicture}[->,>=stealth',shorten >=1pt,auto,node distance=2.5cm,
                    semithick]
  \tikzstyle{every state}=[draw=black,text=black]
	\tikzstyle{every edge} = [draw,thick,-]

  \node[state,fill=red] 	(0)              	{1};
  \node[state,fill=blue]  (1) [above right of=0] 	{2};
  \node[state,fill=yellow] 	(2) [right of=0]  {3};
  \node[state,fill=blue]  (3) [below of=0] 	{4};
  \node[state,fill=yellow] 	(4) [right of=3]  {5};
  \node[state,fill=red]  (5) [right of=2] 	{6};
 

  
  \path[draw=black,thick] 
  		%% drone 1
  		(0)  edge node {} (2)
  		(3)  edge  node {} (4)
  		(5)  edge  node {} (4)
  		(1)  edge  node {} (0)
  		(2)  edge  node {} (3)
  		(1)  edge  node {} (5)
			(2)  edge  node {} (5)
			(3)  edge  node {} (5)
  		;
\end{tikzpicture}
}
\caption[3-Färbung]{3-Färbung}
  \label{Faerbung}
\end{center}
\end{figure}


\subsubsection*{Maximaler Schnitt}

\begin{definition}
Eine Menge $\mathcal{A} = \{A_1,\ldots,A_k\}$ disjunkter Teilmengen einer Menge A ist eine \textcolor{blue}{Partition} von A, wenn $\bigcup \mathcal{A} := \bigcup_{i=1}^k A_i = A$ ist und $A_i \neq \emptyset \quad \forall i$. \cite{Diestel}  
\end{definition}

\begin{definition}
Ist ${V_1,V_2}$ eine Partition von V, so nennen wir die Menge $E(V_1,V_2)$ aller dieser Partitionen verbindenden Kanten von G einen \textcolor{blue}{Schnitt}.  \cite{Diestel} 
\end{definition}

\begin{definition}
Ein \textcolor{blue}{Maximaler Schnitt} ist jener Schnitt $F \neq \emptyset$, wo die Summe der Gewichte der verbindenden Kanten maximal ist. 
\end{definition}


\begin{figure}[htp]
\begin{center}
\resizebox{.5\linewidth}{!}{
\begin{tikzpicture}[->,>=stealth',shorten >=1pt,auto,node distance=2.5cm,
                    semithick]
  \tikzstyle{every state}=[draw=black,text=black]
	\tikzstyle{every edge} = [draw,thick,-]

  \node[state,fill=blue] 	(0)              	{1};
  \node[state,fill=yellow]  (1) [above right of=0] 	{2};
  \node[state,fill=yellow] 	(2) [right of=0]  {3};
  \node[state,fill=blue]  (3) [below of=0] 	{4};
  \node[state,fill=yellow] 	(4) [right of=3]  {5};
  \node[state,fill=yellow]  (5) [right of=2] 	{6};
 

  
  \path[draw=black,thick] 
  		%% drone 1
  		(0)  edge node {5} (2)
  		(3)  edge  node {4} (4)
  		(5)  edge  node {1} (4)
  		(1)  edge  node {1} (0)
  		(2)  edge  node {3} (3)
  		(1)  edge  node {3} (5)
			(2)  edge  node {1} (5)
			(3)  edge  node {2} (5)
  		;
\end{tikzpicture}
}
\caption[Maximaler Schnitt]{Maximaler Schnitt}
  \label{MaxCut}
\end{center}
\end{figure}


\subsection{Polynomdarstellung kombinatorischer Probleme} \label{kombPro}

\subsubsection*{Stable Set}

\begin{lemma}
Sei $G = (V, E)$ ein Graph. Für ein gegebenes $k \in \mathbb{N}$ betrachten wir folgendes polynomielles System:
\begin{align*}
&x_i^2 - x_i = 0 \quad \forall i \in V, \\
&x_i x_j = 0 \quad \forall (i,j) \in E, \\
& \sum_{i \in V} x_i = k.
\end{align*} 

\noindent Dieses System ist genau dann lösbar, wenn $G$ ein Stable Set der Größe k besitzt.
\end{lemma} 

\subsubsection*{k-Färbbarkeit}

\begin{lemma} \label{3color}
Sei $G = (V, E)$ ein Graph. Für ein gegebenes $k \in \mathbb{N}$ betrachten wir folgendes polynomielles System mit $\vert V \vert + \vert E \vert$ Gleichungen:
\begin{align*}
&x_i^k - 1 = 0 \quad \forall i \in V, \\
& \sum_{s = 0}^{k-1} x_i^{k-1-s}x_j^s = 0 \quad \forall (i,j) \in E.
\end{align*} 

\noindent Der Graph G ist genau dann k-färbbar, wenn dieses System eine komplexe Lösung besitzt. Des Weiteren gilt, wenn k ungerade , dann ist G genau dann k-färbbar, wenn das System eine Lösung über $\overline{\mathbb{F}_2}$ besitzt. $\overline{\mathbb{F}_2}$ ist der algebraische Abschluss über dem endlichen Körper mit zwei Elementen.
\end{lemma}

\begin{proof}
Angenommen die Aussage ist wahr über den komplexen Zahlen $\mathbb{C}$. 
\\ $"`\Rightarrow"'$ Sei G k-färbbar, ordne jeder Farbe die k-te Einheitswurzel zu. Sei die j-te Farbe $\beta_j = e^{2\Pi j/k}$; substituiere alle $x_l$ mit der zugehörigen Einheitswurzel der Farbe des l-ten Knotens. \\Also haben wir eine Lösung des Systems: Die Gleichungen $x_i^k-1 = 0$ sind trivialerweise erfüllt. \\Wir betrachten nun die Kantengleichungen: Wir nehmen eine Kante $ij$, da $x_i$ und $x_j$ durch Einheitswurzeln substituiert wurden, gilt $x_i^k - x_j^k = 0$. Des Weiteren gilt: 
\begin{align*}
x_i^k-x_j^k = (x_i-x_j)(x_i^{k-1}+x_i^{k-2}x_j+x_i^{k-3}x_j^2+\ldots+x_j^{k-1}) = 0;
\end{align*}   
Durch die Substitution mit unterschiedlichen Einheitswurzeln gilt $x_i - x_j \neq 0$, also muss der andere Faktor, der den Kantengleichungen entspricht, 0 sein.  
\\ $"`\Leftarrow"'$ Angenommen die Gleichungen seien erfüllt, d.h. der Lösungspunkt muss aus k-ten Einheitswurzeln bestehen. Den benachbarten Knoten müssen verschiedene Einheitswurzeln zugeordnet werden, da: \\
Angenommen einem Paar benachbarter Knoten $ij$ wird die selbe Einheitswurzel zugewiesen. Die Gleichung $x_i^{k-1}+x_i^{k-2}x_j+x_i^{k-3}x_j^2+\ldots+x_j^{k-1} = 0$ wird dann zu $\beta^{k-1}+\beta^{k-1}+\ldots+\beta^{k-1} = k\beta^{k-1} = 0$, jedoch $\beta \neq 0 \lightning$ 
Es verbleibt zu zeigen, dass das gleiche Argument wie wir oben über die komplexen Zahlen gezeigt haben, auch für $\overline{\mathbb{F}_2}$ gilt. Obwohl $x_i^k - 1$ nur eine Nullstelle über $\mathbb{F}_2$ besitzt, nämlich 1, erhält man über $\overline{\mathbb{F}_2}$ verschiedene Nullstellen $1,\beta_i,\ldots,\beta_{k-1}$ (in diesem Fall keine komplexen Zahlen). Damit gilt das gleiche Argument wie oben. Wir müssen nur bei der Rückrichtung aufpassen, da $k\beta_i^{k-1} \neq 0$ sein muss. Diese Bedingung ist jedoch erfüllt, da $k$ ungerade und durch die Konstruktion $\beta_i^k$. 
\end{proof}




\subsubsection*{Maximaler Schnitt}

\begin{lemma}
Sei $G = (V, E)$ ein Graph. Wir können die Menge der Schnitte $SG$ von G als 0-1 Inzidenzvektoren darstellen.

\begin{align*}
&SG := \{ \mathcal{X}^F:F \subseteq E \, ist \, in \, einem \, Schnitt \, von \, G \, enthalten \} \subseteq  \{0,1\}^{\vert E \vert}.
\end{align*} 

\noindent Somit kann der Maximale Schnitt mit den Kantengewichten $w_e \in \mathbb{R}^+$ und $e \in E(G)$ folgend definiert werden:

\begin{align*}
\max\{\sum_{e \in E(G)} w_e x_e : x \in SG\}. 
\end{align*} 

\noindent Die Vektoren $\mathcal{X}^F$ sind Lösungen des polynomiellen Systems 
\begin{align*}
&x_e^2 - x_e = 0 \quad \forall e \in E(G),\\
&\prod_{e\in T\cap E(G)}x_e = 0 \quad \forall T \, ungerader \, Kreis \, in \, G
\end{align*}
\end{lemma}


\section{Hilberts Nullstellensatz} \label{HilbertNull}

Zunächst möchten wir auf die allgemeine Problemdarstellung eingehen. \\
Gegeben: $f_1,\ldots,f_m \in \mathbb{K}[x_1,\ldots,x_n]$ \\
Gesucht: Lösung $x$ für das System $f_1(x) = 0, f_2(x) = 0, \ldots f_m(x) = 0$ (wird auch geschrieben als $F(x) = 0$) \\
Ziel ist es eine Lösung zu diesem System zu finden beziehungsweise zu zeigen, dass es keine Lösung gibt. \\
\\
Bevor wir nun das Theorem für den Hilbertschen Nullstellensatz einführen, möchten wir noch das Fredholm's Alternativtheorem betrachten.

\begin{theorem}[Fredholm's Alternativtheorem] \label{Fredi}
Das Lineare Gleichungssystem (LGS) $Ax=b$ besitzt genau dann eine Lösung, wenn ein Vektor $y$ mit der Eigenschaft $y^TA=0^T$ und $y^Tb\neq 0^T$ existiert. 
\end{theorem}

\noindent Der Hilbertsche Nullstellensatz stellt eine strengere und weitreichendere Verallgemeinerung für nichtlineare Polynomialgleichungen dar.

\begin{theorem}[Hilbert's Nullstellensatz] 
Sei F = $\{f_1,\ldots,f_m\} \subseteq \mathbb{K}[x_1,\ldots,x_n]$.\\
Die Varietät $\{x \in \overline{\mathbb{K}^n} : f_1(x)=0,\ldots,f_s(x)=0\}$ ist genau dann leer, wenn 1 zum Ideal $\left\langle F \right\rangle$, das aus F generiert wurde, gehört. Man beachte $1 \in \left\langle F \right\rangle$ bedeutet, dass Polynome $\beta_1,\ldots,\beta_m$ im Ring $\mathbb{K}[x_1,\ldots,x_n]$ existieren, sodass $1 = \sum_{i=1}^m \beta_i f_i$. Diese polynomielle Identität ist ein Nachweis für die Unlösbarkeit von $F(x) = 0$. 
\end{theorem}

\noindent Wir können leicht erkennen, dass das Fredholm's Theorem eine lineare Variante des Hilbertschen Nullstellensatzes mit linearen Polynomen und Konstanten $\beta_i$'s ist. 

\subsection{Relaxierungen mit Hilfe der Linearen Algebra}

Die Hauptidee besteht darin, das gegebene System in eine Reihe Probleme der Linearen Algebra zu relaxieren und dann diese linearen Probleme zu lösen. \\
Aus dem Hilbertschen Nullstellensatz lässt sich folgendes Korollar ableiten: 
\begin{corollary}
Falls numerische Vektoren $\mu \in \mathbb{K}^m$ existieren, sodass $\sum_{i=1}^m \mu_i f_i = 1 \Rightarrow $ das polynomielle System $F(x) = 0$ ist unlösbar. 
\end{corollary}
\noindent Das Entscheiden, ob ein $\mu \in \mathbb{K}^m$ existiert, sodass $\sum_{i=1}^m \mu_i f_i = 1$ ist nur mehr ein Problem der Linearen Algebra über dem Körper $\mathbb{K}$. \\
Es herrscht ein starkes Zusammenspiel zwischen dem System der nichtlinearen Gleichungen $F(x) = 0$, dem Ideal $\langle F \rangle$ und der Linearen Hülle von F über $\mathbb{K}$. \todo[inline]{definieren in den Grundbegriffen}
Im Folgenden möchten wir auf ein Beispiel eingehen, dass uns zeigt, dass wir nicht immer Unlösbarkeit gezeigt werden kann, auch wenn das System tatsächlich unlösbar ist.
\begin{example} \label{ex1}
Wir betrachten folgende Menge von Polynomen:
\begin{align*}
F := \left\{f_1 := x_1^2-1, f_2 := 2x_1x_2+x_3,f_3:=x_1+x_2,f_4:=x_1+x_3 \right\}
\end{align*}
Wir können zeigen, dass das System $F(x)=0$ unlösbar ist, falls wir ein $\mu \in \mathbb{R}^4$ finden, das die folgende Bedingung erfüllt:
\begin{align*}
&\mu_1 f_1 + \mu_2 f_2 + \mu_3 f_3 + \mu_4 f_4 = 1 \\
\Leftrightarrow \quad &\mu_1(x_1^2-1)+\mu_2(2x_1x_2+x_3)+\mu_3(x_1+x_2)+\mu_4(x_1+x_3)=1 \\
\Leftrightarrow \quad &\mu_1x_1^2+2\mu_2x_1x_2+(\mu_2+\mu_4)x_3+\mu_3x_2+(\mu_3+\mu_4)x_1-\mu_1 = 1
\end{align*}
Mit Hilfe von Koeffizientenvergleich erhalten wir das folgende LGS:
\begin{table}[h]
\begin{center}
\begin{tabular}{rrr}
$-\mu_1=1$ & $\mu_3+\mu_4=0$ & $\mu_3=0$ \\
$\mu_2+\mu_4=0$ & $2\mu_2=0$ & $\mu_1=0$ 
\end{tabular}
\end{center}
\end{table}
\\
$F(x)= 0$ ist zwar nicht lösbar, aber da wir für das obere System auch keine Lösung finden, gibt es somit auch keinen Beweis für die Unlösbarkeit von F(x).

\end{example}

\noindent Um die Anwendung zu vereinfachen, führen wir nun eine Matrixschreibweise ein. Wir konstruieren die Matrix $M_F$, wobei die Spalten die Monome und die Zeilen die Polynome des Systems $F$ repräsentieren. Die Einträge der Matrix entsprechen somit den Koeffizienten der Monome des zugehörigen Polynoms.  Wir definieren den Vektor $\mu := (\mu_1,\mu_2,\ldots,\mu_m)$ und den Vektor $(\textbf{0},1)^T:=(0,\ldots,0,1)^T$ mit genau der Anzahl der Monome an Einträgen. Wir können nun das LGS als $\mu M_F = (\textbf{0},1)^T$ schreiben. 
\begin{note}
Im Fall, dass $F(x)=0$ ein LGS ist, können wir das Theorem \ref{Fredi} anwenden: 
\begin{align*}
F(x)=0 \text{ unlösbar} \Leftrightarrow \mu M_F = (\textbf{0},1)^T \text{ ist lösbar}
\end{align*}
\end{note}  

\begin{example}
Die Matrix $M_F$ zum Beispiel \ref{ex1} sieht folgendermaßen aus: 
\begin{align*}
M_F := \bordermatrix{
	& 1 & x_{1} & x_2 & x_3 & x_1x_2 & x_1^2 \cr
	\mu_1 & -1 & 0 & 0 & 0 & 0 & 1 \cr
	\mu_2 & 0 & 0 & 0 & 1 & 2 & 0 \cr
	\mu_3 & 0 & 1 & 1 & 0 & 0 & 0 \cr
	\mu_4 & 0 & 1 & 0 & 1 & 0 & 0 
}
\end{align*}
\end{example}


\noindent Leider, wie wir bereits im Beispiel \ref{ex1} gesehen haben, können wir keine Aussage über die Lösbarkeit von $F(x) = 0$ treffen, wenn
$\mu M_F = (\textbf{0},1)^T \text{ unlösbar}$. Es besteht jedoch weiterhin die Möglichkeit, auf die Unlösbarkeit von $F(x) = 0$ zu kommen. Der Hilbertsche Nullstellensatz sagt aus, dass durch eine Erweiterung von F durch Polynome vom Ideal von F, das System $\mu M_F = (\textbf{0},1)^T$ lösbar werden kann. Dies erkennt man anhand folgendem Beispiel:

\begin{example}
Das erweiterte System für F aus dem Beispiel \ref{ex1} könnte folgendermaßen aussehen:

\begin{align*}
F'=\{f_1,f_2,f_3,f_4,x_2f_1,x_1f_2,x_1f_3,x_1f_4\}
\end{align*}
Das neue lineare System lässt sich nach dem Koeffizientenvergleich darstellen als:


\begin{table}[h]%
\begin{center}
\begin{tabular}{rrr}
$-\mu_1=1$ & $\mu_3+\mu_4=0$ & $\mu_3-\mu_5=0$ \\
$\mu_2+\mu_4=0$ & $2\mu_2+\mu_7=0$ & $\mu_1+\mu_7+\mu_8=0$ \\
$\mu_6 + \mu_8 = 0$ & $\mu_5+2\mu_6 = 0$
\end{tabular}
\label{}
\end{center}
\end{table}

\noindent Da dieses System die Lösung $\mu = (-1,-\frac{2}{3},-\frac{2}{3},\frac{2}{3},-\frac{2}{3},\frac{1}{3},\frac{4}{3},-\frac{1}{3})$ besitzt, können wir folgern, dass das nichtlineare System $F(x) = 0$ unlösbar ist.
\end{example}

\section{Algorithmus Nullstellensatz in der Linearen Algebra (NulLa)}

Im Folgenden stellen wir einen Algorithmus vor, welcher entscheidet, ob $F(x) = 0$ eine Lösung in $\overline{\mathbb{K}}$ besitzt oder nicht. Die Grundidee des Algorithmus liegt darin, dass überprüft wird, ob $\mu M_F = (\textbf{0},1)^T$ lösbar ist. Das ist genau dann der Fall, wenn $1 \in span_\mathbb{K}(F)$. Falls das System nicht lösbar ist wird $F$ mit Polynomen aus $\langle F \rangle$, der Form $x_if$ für alle $x_i$ und für alle $f \in F$, erweitert und erneut überprüft. Aufgrund der Resultate des Nullstellensatzes wissen wir, dass der Algorithmus terminieren muss, wenn $F(x) = 0$ unlösbar ist. Außerdem erhalten wir auch eine obere Schranke $D$: Wenn $F(x) = 0$ ist unlösbar, dann wissen wir, dass ein Polynom existiert für das folgendes gilt: 
\begin{align*}
\sum_i \beta_i f_i = 1, \qquad deg(\beta_i)\le D
\end{align*}
Daraus folgt, dass dieses Polynom einen Maximalgrad von $deg(F)+D$ besitzt. \\
Das nachfolgende Lemma wird zeigen, dass dieses $D$ eine obere Schranke der Iterationen für unseren Algorithmus bildet.

\begin{lemma} \label{Kollar}
Sei $\mathbb{K}$ ein Körper und $f_1,\ldots,f_k$ Polynome aus $\mathbb{K}[x_1,\ldots,x_n]$ mit Graden $d_1\ge d_2 \ge \ldots \ge d_k \ge 2$. Falls diese Polynome keine gemeinsame Nullstelle über $\overline{\mathbb{K}}$ besitzen, dann existieren $g_1,\ldots,g_k$ in $\mathbb{K}[x_1,\ldots,x_n]$, sodass $\sum_{i=1}^k g_if_i=1, deg(g_if_i) \le D$. $D$ setzt sich wie folgt zusammen:
\begin{align}
D &= \begin{cases} d_1\cdots d_k, & \text{falls $k \le n$}, \\ d_1\cdots d_{n-1}d_k, &
  \text{falls $k > n > 1$}, \\ d_1+d_k-1, & \text{falls $k > n = 1$} \end{cases}
\end{align} 
Diese Abschätzung für $D$ gilt für beliebige Polynome.
\end{lemma}

\noindent Mithilfe des obigen Lemmas können wir nun unseren Algorithmus folgend aufschreiben:\\
\begin{algorithm} [H]
    \SetKwInOut{Input}{Input}
    \SetKwInOut{Output}{Output}

    \underline{function NulLA} $(F,D)$\;
    \Input{Eine endliche Ausgangsmenge von Polynomen $F \subseteq \mathbb{K}[x_1,\ldots,x_n]$ und die maximale Anzahl an Iterationen $D$}
    \Output{\textsc{lösbar}, wenn $F(x) = 0$ ist lösbar über $\overline{\mathbb{K}}$, sonst \textsc{keine Lösung}}
		\For{$k = 0,1,\ldots,D$} {
		\eIf{$1 \in span_\mathbb{K}(F)$}
      {
        \textbf{return} \textsc{keine Lösung}\;
      }
      {
        \textbf{return} Ersetze $span_\mathbb{K}(F)$ durch $(span_\mathbb{K}(F))^+$ (hinzufügen der Polynome $x_iF$ zur Menge $F$)\;
      }
		}
		\textbf{return} \textsc{lösbar}
    \caption{NulLA Algorithmus}
\end{algorithm}

\noindent Falls $F(x)=0$ unlösbar, findet der Algorithmus einen Nachweis dafür nach höchstens D Iterationen. \\
Die tatsächliche Anzahl an Iterationen $r$, die der Algorithmus benötigt, definieren wir nun als Rang von NulLA. Für ein unlösbares System $F(x)=0$ ergibt sich nun ein Nachweis $\sum_{i} \beta_if_i=1$ vom Grad $r + deg(F)$. Allerdings sehen wir durch Lemma \ref{Kollar}, dass die Schranke $D$ exponentielles Wachstum aufweist. Bessere Schranken existieren jedoch für bestimmte kombinatorische Problemstellungen, wie z.B. auch jene, die wir im Abschnitt \ref{kombPro} vorgestellt haben. Anhand des folgenden Korollars sehen wir, dass für diese Probleme eine merkbar bessere Schranke gegenüber der Exponentiellen aus \ref{Kollar} erhalten. 

\begin{corollary}
Für gegebene Polynome $f_1,\ldots,f_s \in \mathbb{K}[x_1,\ldots,x_n]$, wobei $\mathbb{K}$ ein algebraisch abgeschlossener Körper ist und $d = \max\{deg(f_i)\}$, falls $f_1,\ldots,f_s$ keine gemeinsame Nullstelle besitzen (auch nicht im Unendlichen), dann gilt $1 = \sum_{i=1}^s\beta_if_i$ mit $deg(\beta_i) \le n (d-1)$.
\end{corollary}

\noindent Dadurch erhalten wir eine obere Schranke von $2n$ bzw. $n$ für das 3-Färbbarkeitsproblem bzw. Stable Set-Problem. \\

\noindent Das folgende Lemma trifft eine Aussage über die Laufzeit des Algorithmus:
\begin{lemma}\label{polTime}
Sei $d \in \mathbb{Z}_+$ fix und $F = \{f_1,\ldots,f_m\}$ eine Menge von Polynomen aus $\mathbb{K}[x_1,\ldots,x_n]$. Wenn die Bedingungen von NulLA erfüllt sind, dann gilt: Die ersten $d$ Iterationen können in polynomieller Zeit der Größe des Inputs $F$ durchgeführt werden.
\end{lemma}

Obwohl wir nun theoretisch eine polynomielle Laufzeit des Algorithmus gezeigt haben, kann es trotzdem noch Probleme in der praktischen Umsetzung geben. Dies liegt im schnellen Wachstum der Matrizen, die die linearen Systeme repräsentieren, von $O(n^n)$. Glücklicherweise gibt es praktische Resultate, die zeigen, dass das Wachstum des Rangs von NulLA für das Färbbarkeitsproblem oft sehr langsam ist. Dadurch können wir den Algorithmus auch auf große Graphen anwenden, wie wir im folgenden Kapitel sehen werden. \cite{Ausgangsartikel}



\subsection{3-Färbbarkeit und NulLA}

In diesem Kapitel werden wir unseren Algorithmus verwenden, um zu zeigen, dass ein Graph nicht 3-färbbar ist. Wie wir bereits mit Hilfe des Lemmas \ref{3color} gezeigt haben, ist ein Graph genau dann 3-färbbar, wenn das folgende System eine Lösung über $\overline{\mathbb{F}_2}$ besitzt:
\begin{align*}
x_i^3+1=0, \qquad &\forall i \in V(G)\\
x_i^2+x_ix_j+x_j^2=0, \qquad &\forall{i,j}\in E(G) \tag{$\ast$} 
\end{align*}

\noindent Nun können wir ein Korollar zur nicht 3-Färbbarkeit aufstellen.

\begin{corollary}
Ein Graph ist nicht 3-färbbar $\Leftrightarrow$ $\sum_i \beta_if_i = 1, \text{ für } \beta_i \in \mathbb{F}_2[x_1,\ldots,x_n]$ und $f_i \in \mathbb{F}_2[x_1,\ldots,x_n]$ wie in Lemma \ref{3color} definiert. 
\end{corollary}

\noindent Dieses Korollar ermöglicht uns das Operieren über dem Körper $\mathbb{F}_2$, was für praktische Anwendungen sehr zeiteffizient ist. \cite{Loera2011} \\

\noindent Jedoch ist dies nicht für alle Graphen möglich, die nicht 3-Färbbarkeit in angemessener Zeit zu bestimmen, was uns das folgende Lemma zeigt \cite{Loera2009}:

\begin{lemma}
Angenommen $P \not = NP$, dann muss es eine unendlich große Menge von Graphen geben, für welche der Grad des Polynoms zum Nachweis der nicht 3-Färbbarkeit in Abhängigkeit der Anzahl der Knoten und Kanten im Graphen unendlich anwachsen kann.
\end{lemma}

Die nächsten Resultate zeigen uns, für welche Graphen es möglich ist, einen Nachweis für nicht 3-Färbbarkeit mit NulLA und $D=1$ zu erbringen. 

\begin{definition} \label{Polynomdarstellung}
Der NulLa kann einen Nachweis der Unlösbarkeit für $(\ast)$ innerhalb der Schranke $D=1$ erbringen, genau dann wenn Koeffizienten $a_i, a_{ij}, b_{ij},b_{ijk} \in \mathbb{F}$ existieren, sodass:
\begin{align*}
\sum_{i \in V}\left(a_i + \sum_{j\in V}a_{ij}x_j\right)(x_i^3+1)+\sum_{\{i,j\}\in E}\left(b_{ij}+\sum_{k\in V}b_{ijk}x_k\right)(x_i^2+x_ix_j+x_j^2)=1
\end{align*}
\end{definition}

\noindent Nun kommen wir zur kombinatorischen Beschreibung unseres Problems. Wir gehen von einem einfachen ungerichteten Graphen $G=(V,E)$ aus. Nun sei $Arcs(G)$ die Menge die für jede ungerichteten Kante $\{i,j\}\in E(G)$ zwei gerichtete Kanten $(i,j),(j,i)$ enthält:
\begin{align*}
Arcs(G) = \{(i,j):i,j \in V(G), und \{i,j\}\in E(G)\}.
\end{align*}
 
\noindent Mit Hilfe dieser Menge können wir nun zwei Strukturen für Teilgraphen definieren. 

\begin{definition} \label{Subgraphs}
\begin{enumerate}
	\item \textbf{orientiertes partielles Dreieck:} \\
	Gegeben: $\{(i,j),(j,k)\} \subseteq Arcs(G)$ und auch $(k,i) \in Arcs(G)$. Dies induziert einen Kreis der Länge 3 in $G$, deshalb schreiben wir auch $(i,j,k)$.
	\begin{figure}[htp]
\begin{center}
\resizebox{.2\linewidth}{!}{
\begin{tikzpicture}[->,>=stealth',shorten >=1pt,auto,node distance=2.5cm,
                    semithick]
  \tikzstyle{every state}=[draw=black,text=black]
	\tikzstyle{every edge} = [draw,thick]

  \node[state] 	(0)              	{i};
  \node[state]  (1) [above right of=0] 	{j};
  \node[state] 	(2) [below right of=1]  {k};
 

  
  \path[draw=black,thick] 
  		%% drone 1
  		(0)  edge node {} (1)
  		(1)  edge  node {} (2)
  		(2)  edge[draw,thick,dashed]  node {} (0)
  		;
\end{tikzpicture}
}
\caption[orientiertes partielles Dreieck]{orientiertes partielles Dreieck}
  \label{oDreieck}
\end{center}
\end{figure}
	\item \textbf{orientiertes ? Viereck:} \\
	Gegeben: $\{(i,j),(j,k),(k,l),(l,i)\} \subseteq Arcs(G)$ und $(i,k),(j,l) \notin Arcs(G)$. Dies induziert einen Kreis der Länge 4 in $G$, deshalb schreiben wir auch $(i,j,k,l)$. 
	\begin{figure}[htp]
\begin{center}
\resizebox{.2\linewidth}{!}{
\begin{tikzpicture}[->,>=stealth',shorten >=1pt,auto,node distance=2.5cm,
                    semithick]
  \tikzstyle{every state}=[draw=black,text=black]
	\tikzstyle{every edge} = [draw,thick]

  \node[state] 	(0)              	{i};
  \node[state]  (1) [above of=0] 	{j};
  \node[state] 	(2) [right of=1]  {k};
  \node[state]  (3) [below of=2] 	{l};
 

  
  \path[draw=black,thick] 
  		%% drone 1
  		(0)  edge node {} (1)
  		(1)  edge  node {} (2)
  		(2)  edge  node {} (3)
  		(3)  edge  node {} (0)
  		;
\end{tikzpicture}
}
\caption[orientiertes Viereck]{orientiertes Viereck}
  \label{oViereck}
\end{center}
\end{figure}
\item \textbf{ungerades Rad:}\\
Gegeben: Knoten $1,\ldots,n (\quad n \in \mathbb{N}_G)$, wobei Knoten 1 adjazent zu allen anderen Knoten ist und Knoten $i=2,\ldots,n$ adjazent zu Knoten $1, i-1,i+1$. 
	\begin{figure}[htp]
\begin{center}
\resizebox{.2\linewidth}{!}{
\begin{tikzpicture}[->,>=stealth',shorten >=1pt,auto,node distance=2.5cm,
                    semithick]
  \tikzstyle{every state}=[draw=black,text=black]
	\tikzstyle{every edge} = [draw,thick,-]

  \node[state] 	(0)              	{1};
  \node[state]  (1) [left of=0] 	{n};
  \node[state] 	(2) [above left of=0]  {2};
  \node[state]  (3) [above of=0] 	{3};
	\node[state]  (4) [above right of=0] 	{4};
	\node[state]  (5) [right of=0] 	{5};
	\node[state]  (6) [below right of=0] 	{6};
	\node[state]  (7) [below of=0] 	{7};
	
 

  
  \path[draw=black,thick] 
  		%% drone 1
  		(0)  edge node {} (1)
			(0)  edge node {} (2)
			(0)  edge node {} (3)
			(0)  edge node {} (4)
			(0)  edge node {} (5)
			(0)  edge node {} (6)
			(0)  edge node {} (7)
  		(1)  edge[bend left = 10]  node {} (2)
  		(2)  edge[bend left = 10]  node {} (3)
  		(3)  edge[bend left = 10]  node {} (4)
			(4)  edge[bend left = 10]  node {} (5)
			(5)  edge[bend left = 10]  node {} (6)
			(6)  edge[bend left = 10]  node {} (7)
			(7)  edge[bend left = 35,draw,thick,-,dotted]  node {} (1)
  		;
\end{tikzpicture}
}
\caption[ungerades Rad]{ungerades Rad}
  \label{oddWheel}
\end{center}
\end{figure}
\end{enumerate}
\end{definition}

\begin{theorem} \label{3colorTheorem}
Für einen einfachen ungerichteten Graph $G=(V,E)$ sind folgende Aussagen äquivalent:
\begin{enumerate}
	\item Für folgendes polynomielle System über $\mathbb{F}_2$ bringt NulLA einen Nachweis für die nicht 3-Färbbarkeit in D=1
	\begin{align*}
	J_G = \{x_i^3+1=0,x_i^2+x_ix_j+x_j^2=0:i\in V(G),\{i,j\}\in E(G)\}
	\end{align*}
	\item Es existiert eine Menge $C$ von orientierten partiellen Dreiecken und orientierten ? Vierecken aus $Arcs(G)$, sodass 
	\begin{compactenum}[a)]
		\item  $\left|C_{(i,j)}\right|+\left|C_{(j,i)}\right| \equiv 0 \mod{2} \quad \forall \{i,j\} \in E$
		\item $\sum_{(i,j) \in Arcs(G), i < j} \left|C_{(i,j)}\right| \equiv 1 \mod{2}$
	\end{compactenum}
	wobei $C_{(i,j)}$ die Menge der Kreise in C, welche $(i,j) \in Arcs(G)$ enthalten, beschreibt.
\end{enumerate}
Solche Graphen sind nicht 3-färbbar und dies kann in polynomieller Zeit bestimmt werden
\end{theorem}


\noindent In diesem Theorem stellt $C$ eine Kantenüberdeckung mit gerichteten Kanten vom ungerichteten Graphen $G$ dar. Die Bedingung $a)$ sagt aus, dass jede Kante in $G$ von einer geraden Anzahl an gerichteten Kanten aus $C$ überdeckt. Bedingung $b)$ liefert eine Aussage für den Graphen $\widehat{G}$, welcher eine gerichtete Version von $G$ ist, wobei die Richtungen von der Ordnung der Knoten $1<2<\ldots <n$ bestimmt wird. Die Anzahl der Kanten in $\widehat{G}$, die sich auch in $C$ befinden, ist ungerade. Wenn wir nun die Kreise der Länge 3 und der Länge 4 in $G$ betrachten, die den partiellen gerichteten Dreiecken und gerichteten Vierecken entsprechen, so erzeugen diese eine Kantenüberdeckung eines nicht 3-färbbaren Teilgraphen von $G$. Falls ein Graph nun einen nicht 3-färbbaren Teilgraphen enthält, und es für diesen Graphen einen Nachweis für die nicht 3-Färbbarkeit in $D=1$ gibt, dann gibt es den auch für $G$. \\
Die Klasse der Graphen mit dieser Eigenschaft, beinhaltet auch alle Graphen, die ein ungerades Rad enthalten.

\begin{corollary}
Falls ein Graph $G=(V,E)$ ein ungerades Rad als Teilgraph enthält, so gibt es einen Nachweis für nicht 3-Färbbarkeit mit $D=1$.
\end{corollary} 

\begin{proof}
$G$ enthält ein ungerades Grad, wie in Punkt 3 der Definition \ref{Subgraphs} beschrieben, und sei $C$ die Menge der partiellen Dreiecke:
\begin{align*}
C\;:=\;\{(i,1,i+1):2\le i \le n-1\}\cup\{(n,1,2)\}.
\end{align*}
Wie wir in \ref{oddWheel} sehen können, wird jede Kante genau nullmal oder zweimal von partiellen gerichteten Dreiecken von $C$ überdeckt und damit wird Bedingung $a)$ von Theorem \ref{3colorTheorem} erfüllt. Außerdem gilt für jede Kante $(1,i) \in Arcs(G)$, dass sie genau einmal von einem partiellen gerichteten Dreieck aus $C$ abgedeckt wird und die Anzahl der Kanten $(1,i)$ ist ungerade. Dadurch erfüllt $C$ auch die Bedingung $b)$ unseres Theorems. 
\end{proof}

\noindent Ein weiteres nichttriviales Beispiel für den Nachweis der nicht 3-Färbbarkeit des NulLa mit $D=1$ stellt der Grötzsch Graph dar. \todo[inline]{Graph}

\begin{example}
Der Grötzsch Graph enthält wie man sehen kann, keine Kreise der Länge 3. Jedoch enthält er folgende Menge von orientierten Vierecken:
\begin{align*}
C\;:=\;\{(1,2,3,7),(2,3,4,8),(3,4,5,9),(4,5,1,10),(1,10,11,7),(2,6,11,8),(3,7,11,9),(4,8,11,10),(5,9,11,6)\}.
\end{align*}
Auch hier liefert wieder unser Theorem \ref{3colorTheorem} die Aussage dafür, dass für diesen Graphen die nicht 3-Färbbarkeit mit $D=1$ gezeigt werden kann. Jede Kante von $G$ wird genau von zwei Vierecken aus $C$ überdeckt und somit ist Bedingung $a)$ erfüllt. Außerdem kann man durch Abzählen erkennen, dass auch Bedingung $b)$ erfüllt ist. 
\end{example}

\noindent Um das Theorem \ref{3colorTheorem} zu beweisen, benötigen wir 2 Lemmas. Als Vorbereitung dafür, definieren wir zuerst die Menge $H$. Als erstes vereinfachen wir das Polynom zum Nachweis für die nicht 3-Färbbarkeit (siehe Definition \ref{Polynomdarstellung}). Nach Ausmultiplizieren der linken Seite können wir sehen, dass der einzige Koeffizient für den Term $x_jx_i^3$ $a_{ij}$ ist. Daraus folgt, dass $a_{ij} = 0 \forall i,j\in V(G)$. Das gleiche Argument gilt auch für den einzigen Koeffizienten $b_{ij}$ von $x_ix_j$. Dadurch erhält man folgende Vereinfachung:  
\begin{align*}
\sum_{i \in V}a_i(x_i^3+1)+\sum_{\{i,j\}\in E}\left(\sum_{k\in V}b_{ijk}x_k\right)(x_i^2+x_ix_j+x_j^2)=1
\end{align*}

\noindent Nun betrachten wir die Menge der Polynome in $F$:
\begin{align*}
x_i^3+1 \qquad \forall i \in V, \\
x_k(x_i^2+x_ix_j+x_j^2) \qquad \forall \{i,j\} \in E, k \in V.
\end{align*}

Die Elemente von $F$ sind jene Polynome, die einen Nachweis der Unlösbarkeit mit rang 1 darstellen.  Ein solcher Nachweis kann nur erbracht werden, genau dann wenn das konstante Polynom $1$ in der linearen Hülle von $F$ enthalten ist: $1 \in \langle F \rangle_{\mathbb{F}_2}, \text{ wobei} \langle F \rangle_{\mathbb{F}_2} \text{ Vektorraum über } \mathbb{F}_2$, der durch die Polynome von $F$ generiert wird. 
Nun vereinfachen wir die Menge $F$ und definieren sie als unsere Menge $H$:
\begin{table}%
\begin{center}
\begin{tabular}{lr}
$x_i^2x_j +x_ix_j^2+1$ & $\forall \{i,j\} \in E,$ \\
$x_ix_j^2+x_jx_k^2$ & $\forall (i,j),(j,k),(k,i) \in Arcs(G),$ \\
$x_ix_j^2+x_jx_k^2+x_kx_l^2+x_lx_i^2$ & $\qquad \forall (i,j),(j,k),(k,l),(l,i) \in Arcs(G),$ \\
& $(i,k),(j,l) \notin Arcs(G).$
\end{tabular}
\end{center}
\end{table}

Wenn wir die Monome $x_ix_j^2$ als gerichtete Kanten $(i,j)$ betrachten, beschreiben die Polynome in der zweiten Zeile unsere orientierten partiellen Dreiecke und jene in Zeile drei die orientierten Vierecke. Das erste Lemma, welches wir zum Beweis für unser Theorem benötigen, sagt aus, dass wir $H$ anstatt von $F$ verwenden können.

\begin{lemma}
$1 \in \langle F \rangle_{\mathbb{F}_2} \Leftrightarrow 1 \in \langle H \rangle_{\mathbb{F}_2}$ 
\end{lemma}

Im folgenden Lemma wird beschrieben, dass sowohl die Bedingungen von Theorem \ref{3colorTheorem} als auch die Bedingung $1 \in \langle H \rangle_{\mathbb{F}_2}$ zum Nachweis der nicht 3-Färbbarkeit genügen.

\begin{lemma}
Es existiert eine Menge $C$ von gerichteten partiellen Dreiecken und orientierten Vierecken, die die Bedingungen $a)$ und $b)$ des Theorems \ref{3colorTheorem} erfüllen $
\Leftrightarrow 1 \in \langle H \rangle_{\mathbb{F}_2}$ 
\end{lemma}

\noindent Aus der Kombination dieser beiden Lemmas sehen wir nun, dass sie einen Beweis für Theorem \ref{3colorTheorem} bilden. Für die beiden Beweise verweisen wir auf \cite{Ausgangsartikel}. Das die Entscheidung, ob solche Graphen nicht 3-färbbar sind, in polynomieller Zeit erfolgt, zeigt uns Lemma \ref{polTime}.  


\subsection{Stable Sets und NulLA}


\begin{definition} \label{StableSetPolynom}
Ein Graph besitzt kein Stable Set der Größe $\ge \alpha (G)$, wenn es eine Lösung für das folgende System gibt:
\begin{align*}
1 = A\left(-(\alpha(G) + r) + \sum_{i=1}^nx_i\right)+\sum_{k \in V(G)} Q_k(x_k^2-x_k)+\sum_{\{u,v\}\in E(G)} Q_{uv}(x_ux_v) \qquad r \ge 1. 
\end{align*}
\end{definition}

\begin{theorem}
Für einen Graph $G$ existiert ein Nachweis des Nullstellensatzes mit Grad $\alpha(G)$, für die Nichtexistenz eines Stable Sets, der Größe $\alpha(G) +r (r\ge 1), $ sodass die Definition \ref{StableSetPolynom} erfüllt ist, \\
wobei 
\begin{align*}
\end{align*}
\end{theorem}


\noindent Zum Schluss können wir zusammenfassen, dass die Polynome, die zum Verifizieren des Nullstellensatzes dienen, sehr dicht sind, da alle quadratfreien Monome, welche Stable Sets darstellen, in ihnen vorkommen.
Das stellt jedoch ein großes Hindernis für die Berechnung dar. In diesem Fall ist es so, dass die Berechnung von Hilberts Nullstellensatz zumindest so schwer ist wie das Zählen aller möglichen Stable Sets eines Graphen. Dieses Problem ist also bereits für Graphen mit geringen Knotengrad $\#P$-vollständig. \todo[inline]{\#P,NP,P definieren}




\section{Vergleich mit anderen Algorithmen zur 3-Färbbarkeit}



\section{Beweis Nullstellensatz}

%%% Local Variables: 
%%% mode: latex
%%% TeX-master: "master"
%%% End: 

\include{Wermachtwas}


\listoffigures
%\addcontentsline{toc}{chapter}{Literaturverzeichnis} 
\bibliographystyle{abbrvnat}
%\renewcommand{\bibfont}{\normalsize} 
\bibliography{akp}
% rendl: general purpose huge bib-file


\end{document}


%%% Local Variables:  
%%% mode: latex 
%%% TeX-master: t 
%%% End: \documentclass[10pt,a4paper]{article}
