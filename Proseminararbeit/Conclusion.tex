\part*{Zusammenfassung}

In dieser Arbeit haben wir uns mit einem Resultat aus der Algebraischen Geometrie befasst, welches eine Lösungsmöglichkeit für kombinatorische Optimierungsprobleme darstellt. Genauer gesagt, haben wir uns mit dem Hilbert'schen Nullstellensatz einhergehend befasst und haben seine Fähigkeiten zur Entscheidbarkeit von 3--Färbbarkeit und Stable Sets genauer betrachtet. Der konkrete Algorithmus wird NulLA genannt und wir konnten erkennen, dass es für diesen ein Leichtes ist, verschiedene Probleminstanzen auf Nicht 3--Färbbarkeit zu überprüfen. Allerdings stellte sich heraus, dass bereits kleine Graphen, wie z.B. der Turán--Graph, NulLA Schwierigkeiten bereiten. \\
Außerdem haben wir noch weitere ähnliche Lösungsmethoden betrachtet und stellten Vergleiche mit den verschiedenen Algorithmen auf. Dadurch konnten wir feststellen, dass der Hilbert'sche Nullstellensatz mit den besten bisher entwickelten Algorithmen sehr gut mithalten kann. \\
Nach Betrachtung wichtiger Resultate aus der Diskreten Mathematik, konnten wir unsere Arbeit mit dem Beweis des Hilbert'schen Nullstellensatz vollenden.

\todo[inline]{Konsitenz in Schreibweise prüfen! z.B. Hilbert'schen oder Hilbert's}

%%% Local Variables: 
%%% mode: latex
%%% TeX-master: "master"
%%% End: 
