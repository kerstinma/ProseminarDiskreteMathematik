\vspace*{0.2cm}\section*{Abstract}
In dieser Proseminararbeit widmen wir uns dem Hilbert'schen Nullstellensatz, der zu den klassischen Theoremen in der Algebraischen Geometrie zählt. Im Fokus steht die Verwendung des Nullstellensatzes zum Beweisen der Unlösbarkeit von kombinatorischen Optimierungsproblemen, wie zum Beispiel der 3-Färbbarkeit eines Graphen. \\

\noindent Zu Beginn werden wir auf Grundbegriffe aus der Graphentheorie eingehen. Danach wird der Nullstellensatz und seine Anwendungen in der Diskreten Optimierung genauer behandelt. Zum Schluss widmen wir uns noch theoretischen Resultaten und dem Beweis des Nullstellensatzes. 

%%% Local Variables: 
%%% mode: latex
%%% TeX-master: "master"
%%% End: 
