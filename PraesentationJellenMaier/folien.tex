\subsection{Einleitung}
\begin{mslide}{Einleitung}

\begin{definition}
Die \textcolor{blue}{k-Färbung} eines Graphen ist eine Abbildung $f: V(G) \rightarrow S$ mit $xy \in E(G): f(x) \neq f(y)$, wobei S Menge der Farben und $\vert S \vert = k$. 
\end{definition}

\begin{figure}[htp]
\begin{center}
\resizebox{.4\linewidth}{!}{
\begin{tikzpicture}[->,>=stealth',shorten >=1pt,auto,node distance=2.5cm,
                    semithick]
  \tikzstyle{every state}=[draw=black,text=black]
	\tikzstyle{every edge} = [draw,thick,-]

  \node[state,fill=red] 	(0)              	{1};
  \node[state,fill=blue]  (1) [above right of=0] 	{2};
  \node[state,fill=yellow] 	(2) [right of=0]  {3};
  \node[state,fill=blue]  (3) [below of=0] 	{4};
  \node[state,fill=yellow] 	(4) [right of=3]  {5};
  \node[state,fill=red]  (5) [right of=2] 	{6};
 

  
  \path[draw=black,thick] 
  		%% drone 1
  		(0)  edge node {} (2)
  		(3)  edge  node {} (4)
  		(5)  edge  node {} (4)
  		(1)  edge  node {} (0)
  		(2)  edge  node {} (3)
  		(1)  edge  node {} (5)
			(2)  edge  node {} (5)
			(3)  edge  node {} (5)
  		;
\end{tikzpicture}
}
\end{center}
\end{figure}

\framebreak 

Sei $G = (V, E)$ ein Graph
\begin{align*}
&x_i^k - 1 = 0 \quad \forall i \in V, \\
& \sum_{s = 0}^{k-1} x_i^{k-1-s}x_j^s = 0 \quad \forall (i,j) \in E.
\end{align*} 

\begin{thm}
Graph G ist k-färbbar $\Leftrightarrow$ System besitzt eine komplexe Lösung
\end{thm}
\begin{thm}
G ist k-färbbar und k ungerade $\Leftrightarrow$ System besitzt gemeinsame Nullstelle über $\overline{\mathbb{F}_2}$, wobei  $\overline{\mathbb{F}_2}$ algebraische Abschluss über dem endlichen Körper mit zwei Elementen
\end{thm}
\nocite{*}

\framebreak

\begin{proof}
Angenommen die Aussage ist wahr über den komplexen Zahlen $\mathbb{C}$. \\
 "`$\Rightarrow$"' Sei G k-färbbar, ordne jeder Farbe die k-te Einheitswurzel zu. Sei die j-te Farbe $\beta_j = e^{2\Pi j/k}$; substituiere alle $x_l$ mit der zugehörigen Einheitswurzel der Farbe des l-ten Knotens. Also haben wir eine Lösung des Systems: Die Gleichungen $x_i^k-1 = 0$ sind trivialerweise erfüllt. \\Wir betrachten nun die Kantengleichungen: Wir nehmen eine Kante $ij$, da $x_i$ und $x_j$ durch Einheitswurzeln substituiert wurden, gilt $x_i^k - x_j^k = 0$. Des Weiteren gilt: 
\begin{align*}
x_i^k-x_j^k = (x_i-x_j)(x_i^{k-1}+x_i^{k-2}x_j+x_i^{k-3}x_j^2+\ldots+x_j^{k-1}) = 0;
\end{align*}   
Durch die Substitution mit unterschiedlichen Einheitswurzeln gilt $x_i - x_j \neq 0$, also muss der andere Faktor, der den Kantengleichungen entspricht, 0 sein.
\end{proof}

\framebreak 

\begin{proof}
"`$\Leftarrow$"' Angenommen die Gleichungen seien erfüllt, d.h. der Lösungspunkt muss aus k-ten Einheitswurzeln bestehen. Den benachbarten Knoten müssen verschiedene Einheitswurzeln zugeordnet werden, da: \\
Angenommen einem Paar benachbarter Knoten $ij$ wird die selbe Einheitswurzel zugewiesen. Die Gleichung $x_i^{k-1}+x_i^{k-2}x_j+x_i^{k-3}x_j^2+\ldots+x_j^{k-1} = 0$ wird dann zu $\beta^{k-1}+\beta^{k-1}+\ldots+\beta^{k-1} = k\beta^{k-1} = 0$, jedoch $\beta \neq 0 \lightning$ 
\end{proof}

\framebreak

\begin{definition}
Paarweise nicht benachbarte Knoten oder Kanten nennt man \textcolor{blue}{unabhängig}. Eine Teilmenge von V oder E heißt unabhängig, wenn ihre Elemente paarweise nicht benachbart sind. Unabhängige Knotenmengen nennt man auch \textcolor{blue}{stabil}. 
\end{definition}

\begin{definition}
Ein \textcolor{blue}{Stable Set} ist eine unabhängige Knotenmenge.  $\alpha (G)$ ist die maximale Größe eines Stable Sets vom Graph $G$.
\end{definition}

\begin{figure}[h]
\begin{center}
\resizebox{.25\linewidth}{!}{
\begin{tikzpicture}[->,>=stealth',shorten >=1pt,auto,node distance=2.5cm,
                    semithick]
  \tikzstyle{every state}=[draw=black,text=black]
	\tikzstyle{every edge} = [draw,thick,-]

  \node[state] 	(0)              	{1};
  \node[state,fill=red]  (1) [above right of=0] 	{2};
  \node[state,fill=red] 	(2) [right of=0]  {3};
  \node[state]  (3) [below of=0] 	{4};
  \node[state,fill=red] 	(4) [right of=3]  {5};
  \node[state]  (5) [right of=2] 	{6};
 

  
  \path[draw=black,thick] 
  		%% drone 1
  		(0)  edge node {} (2)
  		(3)  edge  node {} (4)
  		(5)  edge  node {} (4)
  		(1)  edge  node {} (0)
  		(2)  edge  node {} (3)
  		(1)  edge  node {} (5)
			(2)  edge  node {} (5)
			(3)  edge  node {} (5)
  		;
\end{tikzpicture}
}
\end{center}
\end{figure}

\framebreak

Sei $G = (V, E)$ ein Graph. Für ein gegebenes $k \in \mathbb{N}$ betrachten wir folgendes polynomielles System:
\begin{align*}
&x_i^2 - x_i = 0 \quad \forall i \in V, \\
&x_i x_j = 0 \quad \forall (i,j) \in E, \\
& \sum_{i \in V} x_i = k.
\end{align*} 

\begin{thm}
System ist lösbar $\Leftrightarrow$ $G$ besitzt Stable Set der Größe k  
\end{thm}

\end{mslide}

\subsection{Hilberts Nullstellensatz}
\begin{mslide}{Hilberts Nullstellensatz}

\begin{block}{Problemdarstellung}
Gegeben: $f_1,\ldots,f_m \in \mathbb{K}[x_1,\ldots,x_n]$ \\
Gesucht: Lösung $x$ für das System $f_1(x) = 0, f_2(x) = 0, \ldots f_m(x) = 0$ (wird auch geschrieben als $F(x) = 0$) \\
Ziel ist es eine Lösung zu diesem System zu finden beziehungsweise zu zeigen, dass es keine Lösung gibt. \\
\end{block}


\begin{thm}[Fredholm's Alternativtheorem] \label{Fredi}
Das Lineare Gleichungssystem $Ax=b$ besitzt genau dann eine Lösung, wenn ein Vektor $y$ mit der Eigenschaft $y^TA=0^T$ und $y^Tb\neq 0^T$ existiert. 
\end{thm}

\framebreak

\begin{thm}[Hilbert's Nullstellensatz] 
Sei F = $\{f_1,\ldots,f_m\} \subseteq \mathbb{K}[x_1,\ldots,x_n]$.\\
Die Varietät $\{x \in \overline{\mathbb{K}^n} : f_1(x)=0,\ldots,f_s(x)=0\}$ ist genau dann leer, wenn 1 zum Ideal $\left\langle F \right\rangle$, das aus F generiert wurde, gehört. Man beachte $1 \in \left\langle F \right\rangle$ bedeutet, dass Polynome $\beta_1,\ldots,\beta_m$ im Ring $\mathbb{K}[x_1,\ldots,x_n]$ existieren, sodass $1 = \sum_{i=1}^m \beta_i f_i$. Diese polynomielle Identität ist ein Nachweis für die Unlösbarkeit von $F(x) = 0$. 
\end{thm}

\begin{kor}
Falls numerische Vektoren $\mu \in \mathbb{K}^m$ existieren, sodass $\sum_{i=1}^m \mu_i f_i = 1 \Rightarrow $ das polynomielle System $F(x) = 0$ ist unlösbar. 
\end{kor}

\framebreak

\begin{ex} \label{ex1}
Wir betrachten folgende Menge von Polynomen:
\begin{align*}
F := \left\{f_1 := x_1^2-1, f_2 := 2x_1x_2+x_3,f_3:=x_1+x_2,f_4:=x_1+x_3 \right\}
\end{align*}
Wir können zeigen, dass das System $F(x)=0$ unlösbar ist, falls wir ein $\mu \in \mathbb{R}^4$ finden, das die folgende Bedingung erfüllt:
\begin{align*}
&\mu_1 f_1 + \mu_2 f_2 + \mu_3 f_3 + \mu_4 f_4 = 1 \\
\Leftrightarrow \quad &\mu_1(x_1^2-1)+\mu_2(2x_1x_2+x_3)+\mu_3(x_1+x_2)+\mu_4(x_1+x_3)=1 \\
\Leftrightarrow \quad &\mu_1x_1^2+2\mu_2x_1x_2+(\mu_2+\mu_4)x_3+\mu_3x_2+(\mu_3+\mu_4)x_1-\mu_1 = 1
\end{align*}
Mit Hilfe von Koeffizientenvergleich erhalten wir das folgende LGS:
\begin{table}[h]
\begin{center}
\begin{tabular}{rrr}
$-\mu_1=1$ & $\mu_3+\mu_4=0$ & $\mu_3=0$ \\
$\mu_2+\mu_4=0$ & $2\mu_2=0$ & $\mu_1=0$ 
\end{tabular}
\end{center}
\end{table}
%$F(x)= 0$ ist zwar nicht lösbar, aber da wir für das obere System auch keine Lösung finden, gibt es somit auch keinen Beweis für die Unlösbarkeit von F(x).
\end{ex}

\framebreak

Im Fall, dass $F(x)=0$ ein LGS ist, können wir das Theorem \ref{Fredi} anwenden: 
\begin{align*}
F(x)=0 \text{ unlösbar} \Leftrightarrow \mu M_F = (\textbf{0},1)^T \text{ ist lösbar}
\end{align*} 

\begin{ex}
Die Matrix $M_F$ zum Beispiel \ref{ex1} sieht folgendermaßen aus: 
\begin{align*}
M_F := \bordermatrix{
	& x_{1} & x_2 & x_3 & x_1x_2 & x_1^2 & 1 \cr
	\mu_1 & 0 & 0 & 0 & 0 & 1 & -1 \cr
	\mu_2 & 0 & 0 & 1 & 2 & 0 & 0 \cr
	\mu_3 & 1 & 1 & 0 & 0 & 0 & 0 \cr
	\mu_4 & 1 & 0 & 1 & 0 & 0 & 0
}
\end{align*}
\end{ex}

\framebreak

Der Hilbertsche Nullstellensatz sagt aus, dass durch eine Erweiterung von F durch Polynome vom Ideal von F, das System $\mu M_F = (\textbf{0},1)^T$ lösbar werden kann.

\begin{ex}
Das erweiterte System für F aus dem Beispiel \ref{ex1} könnte folgendermaßen aussehen:

\begin{align*}
F'=\{f_1,f_2,f_3,f_4,x_2f_1,x_1f_2,x_1f_3,x_1f_4\}
\end{align*}
Das neue lineare System lässt sich nach dem Koeffizientenvergleich darstellen als:


\begin{table}[h]%
\begin{center}
\begin{tabular}{rrr}
$-\mu_1=1$ & $\mu_3+\mu_4=0$ & $\mu_3-\mu_5=0$ \\
$\mu_2+\mu_4=0$ & $2\mu_2+\mu_7=0$ & $\mu_1+\mu_7+\mu_8=0$ \\
$\mu_6 + \mu_8 = 0$ & $\mu_5+2\mu_6 = 0$
\end{tabular}
\label{}
\end{center}
\end{table}

\noindent Da dieses System die Lösung $\mu = (-1,-\frac{2}{3},-\frac{2}{3},\frac{2}{3},-\frac{2}{3},\frac{1}{3},\frac{4}{3},-\frac{1}{3})$ besitzt, können wir folgern, dass das nichtlineare System $F(x) = 0$ unlösbar ist.
\end{ex}

\end{mslide}

\subsection{Algorithmus Nullstellensatz in der Linearen Algebra (NulLa)}
\begin{mslide}{Algorithmus Nullstellensatz in der Linearen Algebra (NulLa)}

\textbf{Grundidee des Algorithmus:} 
\begin{itemize}
\item Überprüfung, ob $\mu M_F = (\textbf{0},1)^T$ lösbar $\Leftrightarrow$ $1 \in span_\mathbb{K}(F)$ 
\item System nicht lösbar $\Rightarrow$ $F$ wird mit Polynomen aus $\langle F \rangle$, der Form $x_if$ für alle $x_i$ und für alle $f \in F$, erweitert und erneut überprüft 
\item Algorithmus terminiert $\Leftrightarrow$ $F(x) = 0$ unlösbar 
\item Obere Schranke $D$: Wenn $F(x) = 0$ ist unlösbar, dann existiert Polynom für das folgendes gilt: 
\begin{align*}
\sum_i \beta_i f_i = 1, \qquad deg(\beta_i)\le D
\end{align*}
Dieses Polynom besitzt Maximalgrad von $deg(F)+D$.
\end{itemize}

\begin{lem} \label{Kollar}
Sei $\mathbb{K}$ ein Körper und $f_1,\ldots,f_k$ Polynome aus $\mathbb{K}[x_1,\ldots,x_n]$ mit Graden $d_1\ge d_2 \ge \ldots \ge d_k \ge 2$. Falls diese Polynome keine gemeinsame Nullstelle über $\overline{\mathbb{K}}$ besitzen, dann existieren $g_1,\ldots,g_k$ in $\mathbb{K}[x_1,\ldots,x_n]$, sodass $\sum_{i=1}^k g_if_i=1, deg(g_if_i) \le D$. $D$ setzt sich wie folgt zusammen:
\begin{align*}
D &= \begin{cases} d_1\cdots d_k, & \text{falls $k \le n$}, \\ d_1\cdots d_{n-1}d_k, &
  \text{falls $k > n > 1$}, \\ d_1+d_k-1, & \text{falls $k > n = 1$} \end{cases}
\end{align*} 
Diese Abschätzung für $D$ gilt für beliebige Polynome.
\end{lem}

\framebreak

\begin{center}
\resizebox{.9\linewidth}{!}{
\begin{algorithm} [H]
    \SetKwInOut{Input}{Input}
    \SetKwInOut{Output}{Output}

    \underline{function NulLA} $(F,D)$\;
    \Input{Eine endliche Ausgangsmenge von Polynomen $F \subseteq \mathbb{K}[x_1,\ldots,x_n]$ und die maximale Anzahl an Iterationen $D$}
    \Output{\textsc{lösbar}, wenn $F(x) = 0$ ist lösbar über $\overline{\mathbb{K}}$, sonst \textsc{keine Lösung}}
		\For{$k = 0,1,\ldots,D$} {
		\eIf{$1 \in span_\mathbb{K}(F)$}
      {
        \textbf{return} \textsc{keine Lösung}\;
      }
      {
        \textbf{return} Ersetze $span_\mathbb{K}(F)$ durch $(span_\mathbb{K}(F))^+$ (hinzufügen der Polynome $x_iF$ zur Menge $F$)\;
      }
		}
		\textbf{return} \textsc{lösbar}
    \caption{NulLA Algorithmus}
\end{algorithm}
}
\end{center}

\framebreak

\begin{kor}[lineare Schranke]
Für gegebene Polynome $f_1,\ldots,f_s \in \mathbb{K}[x_1,\ldots,x_n]$, wobei $\mathbb{K}$ ein algebraisch abgeschlossener Körper ist und $d = \max\{deg(f_i)\}$, falls $f_1,\ldots,f_s$ keine gemeinsame Nullstelle besitzen (auch nicht im Unendlichen), dann gilt $1 = \sum_{i=1}^s\beta_if_i$ mit $deg(\beta_i) \le n (d-1)$.
\end{kor}


\begin{lem}[Laufzeit von NulLA]\label{polTime}
Sei $d \in \mathbb{Z}_+$ fix und $F = \{f_1,\ldots,f_m\}$ eine Menge von Polynomen aus $\mathbb{K}[x_1,\ldots,x_n]$. Wenn die Bedingungen von NulLA erfüllt sind, dann gilt: Die ersten $d$ Iterationen können in polynomieller Zeit der Größe des Inputs $F$ durchgeführt werden.
\end{lem}
\end{mslide}



\subsection{3-Färbbarkeit und NulLA}
\begin{mslide}{3-Färbbarkeit und NulLA}

Graph 3-färbbar $\Leftrightarrow$ System besitzt Lösung über $\overline{\mathbb{F}_2}$:
\begin{align*}
x_i^3+1=0, \qquad &\forall i \in V(G)\\
x_i^2+x_ix_j+x_j^2=0, \qquad &\forall{i,j}\in E(G) \tag{$\ast$} 
\end{align*}

\begin{kor}
Ein Graph ist nicht 3-färbbar $\Leftrightarrow$ $\sum_i \beta_if_i = 1, \text{ für } \beta_i \in \mathbb{F}_2[x_1,\ldots,x_n]$ und $f_i \in \mathbb{F}_2[x_1,\ldots,x_n]$ wie in  $(\ast)$ definiert. 
\end{kor}

\begin{lem}
Angenommen $P \not = NP$, dann muss es eine unendlich große Menge von Graphen geben, für welche der Grad des Polynoms zum Nachweis der nicht 3-Färbbarkeit in Abhängigkeit der Anzahl der Knoten und Kanten im Graphen unendlich anwachsen kann.
\end{lem}

\framebreak

\begin{definition} \label{Polynomdarstellung}
Der NulLa kann einen Nachweis der Unlösbarkeit für $(\ast)$ innerhalb der Schranke $D=1$ erbringen, genau dann wenn Koeffizienten $a_i, a_{ij}, b_{ij},b_{ijk} \in \mathbb{F}$ existieren, sodass:
\begin{align*}
\sum_{i \in V}\left(a_i + \sum_{j\in V}a_{ij}x_j\right)(x_i^3+1)+\sum_{\{i,j\}\in E}\left(b_{ij}+\sum_{k\in V}b_{ijk}x_k\right)(x_i^2+x_ix_j+x_j^2)=1
\end{align*}
\end{definition}

\framebreak

\textbf{Kombinatorische Beschreibung des Problems:} \\
Ausgangspunkt: einfacher ungerichteter Graph $G=(V,E)$ 
\begin{align*}
Arcs(G) := \{(i,j):i,j \in V(G), und \{i,j\}\in E(G)\}.
\end{align*}

\begin{definition}[Strukturen für Teilgraphen I] 
 
\textbf{orientiertes partielles Dreieck:} 
	Gegeben: $\{(i,j),(j,k)\} \subseteq Arcs(G)$ und auch $(k,i) \in Arcs(G)$. Dies induziert einen Kreis der Länge 3 in $G$, deshalb schreiben wir auch $(i,j,k)$.
	\begin{figure}[htp]
\begin{center}
\resizebox{.2\linewidth}{!}{
\begin{tikzpicture}[->,>=stealth',shorten >=1pt,auto,node distance=2.5cm,
                    semithick]
  \tikzstyle{every state}=[draw=black,text=black]
	\tikzstyle{every edge} = [draw,thick]

  \node[state] 	(0)              	{i};
  \node[state]  (1) [above right of=0] 	{j};
  \node[state] 	(2) [below right of=1]  {k};
 

  
  \path[draw=black,thick] 
  		%% drone 1
  		(0)  edge node {} (1)
  		(1)  edge  node {} (2)
  		(2)  edge[draw,thick,dashed]  node {} (0)
  		;
\end{tikzpicture}
}
  \label{oDreieck}
\end{center}
\end{figure}
\end{definition}

\framebreak

\begin{definition}[Strukturen für Teilgraphen III]
	\textbf{orientiertes ? Viereck:} \\
	Gegeben: $\{(i,j),(j,k),(k,l),(l,i)\} \subseteq Arcs(G)$ und $(i,k),(j,l) \notin Arcs(G)$. Dies induziert einen Kreis der Länge 4 in $G$, deshalb schreiben wir auch $(i,j,k,l)$. 
	\begin{figure}[htp]
\begin{center}
\resizebox{.2\linewidth}{!}{
\begin{tikzpicture}[->,>=stealth',shorten >=1pt,auto,node distance=2.5cm,
                    semithick]
  \tikzstyle{every state}=[draw=black,text=black]
	\tikzstyle{every edge} = [draw,thick]

  \node[state] 	(0)              	{i};
  \node[state]  (1) [above of=0] 	{j};
  \node[state] 	(2) [right of=1]  {k};
  \node[state]  (3) [below of=2] 	{l};
 

  
  \path[draw=black,thick] 
  		%% drone 1
  		(0)  edge node {} (1)
  		(1)  edge  node {} (2)
  		(2)  edge  node {} (3)
  		(3)  edge  node {} (0)
  		;
\end{tikzpicture}
}
  \label{oViereck}
\end{center}
\end{figure}
\end{definition}
\framebreak
\begin{definition}[Strukturen für Teilgraphen III]
\textbf{ungerades Rad:}\\
Gegeben: Knoten $1,\ldots,n \quad (n \in \mathbb{N}_G)$, wobei Knoten 1 adjazent zu allen anderen Knoten ist und Knoten $i=2,\ldots,n$ adjazent zu Knoten $1, i-1,i+1$. 
	\begin{figure}[htp]
\begin{center}
\resizebox{.2\linewidth}{!}{
\begin{tikzpicture}[->,>=stealth',shorten >=1pt,auto,node distance=2.5cm,
                    semithick]
  \tikzstyle{every state}=[draw=black,text=black]
	\tikzstyle{every edge} = [draw,thick,-]

  \node[state] 	(0)              	{1};
  \node[state]  (1) [left of=0] 	{n};
  \node[state] 	(2) [above left of=0]  {2};
  \node[state]  (3) [above of=0] 	{3};
	\node[state]  (4) [above right of=0] 	{4};
	\node[state]  (5) [right of=0] 	{5};
	\node[state]  (6) [below right of=0] 	{6};
	\node[state]  (7) [below of=0] 	{7};
	
 

  
  \path[draw=black,thick] 
  		%% drone 1
  		(0)  edge node {} (1)
			(0)  edge node {} (2)
			(0)  edge node {} (3)
			(0)  edge node {} (4)
			(0)  edge node {} (5)
			(0)  edge node {} (6)
			(0)  edge node {} (7)
  		(1)  edge[bend left = 10]  node {} (2)
  		(2)  edge[bend left = 10]  node {} (3)
  		(3)  edge[bend left = 10]  node {} (4)
			(4)  edge[bend left = 10]  node {} (5)
			(5)  edge[bend left = 10]  node {} (6)
			(6)  edge[bend left = 10]  node {} (7)
			(7)  edge[bend left = 35,draw,thick,-,dotted]  node {} (1)
  		;
\end{tikzpicture}
}
  \label{oddWheel}
\end{center}
\end{figure}
\end{definition}

\framebreak

\begin{thm} \label{3colorTheorem}
Für einen einfachen ungerichteten Graph $G=(V,E)$ sind folgende Aussagen äquivalent:
\begin{compactenum}[1.]
	\item Für folgendes polynomielle System über $\mathbb{F}_2$ bringt NulLA einen Nachweis für die nicht 3-Färbbarkeit in D=1
	\begin{align*}
	J_G = \{x_i^3+1=0,x_i^2+x_ix_j+x_j^2=0:i\in V(G),\{i,j\}\in E(G)\}
	\end{align*}
	\item Es existiert eine Menge $C$ von orientierten partiellen Dreiecken und orientierten ? Vierecken aus $Arcs(G)$, sodass 
	\begin{compactenum}[a)]
		\item  $\left|C_{(i,j)}\right|+\left|C_{(j,i)}\right| \equiv 0 \mod{2} \quad \forall \{i,j\} \in E$
		\item $\sum_{(i,j) \in Arcs(G), i < j} \left|C_{(i,j)}\right| \equiv 1 \mod{2}$
	\end{compactenum}
	wobei $C_{(i,j)}$ die Menge der Kreise in C, welche $(i,j) \in Arcs(G)$ enthalten, beschreibt.
\end{compactenum}
Solche Graphen sind nicht 3-färbbar und dies kann in polynomieller Zeit bestimmt werden
\end{thm}

\framebreak

\begin{kor}
Falls ein Graph $G=(V,E)$ ein ungerades Rad als Teilgraph enthält, so gibt es einen Nachweis für nicht 3-Färbbarkeit mit $D=1$.
\end{kor} 

\begin{proof}
$G$ enthält ein ungerades Grad, und sei $C$ die Menge der partiellen Dreiecke:
\begin{align*}
C\;:=\;\{(i,1,i+1):2\le i \le n-1\}\cup\{(n,1,2)\}.
\end{align*}
%Jede Kante wird genau nullmal oder zweimal von partiellen gerichteten Dreiecken von $C$ überdeckt $\Rightarrow$ Bedingung $a)$ erfüllt \\
%Jede Kante $(1,i) \in Arcs(G)$ wird genau einmal von einem partiellen gerichteten Dreieck $(i < j, \forall i,j \in E(G))$ aus $C$ abgedeckt und die Anzahl der Kanten $(1,i)$ ist ungerade $\Rightarrow$ Bedingung $b)$ erfüllt
	\begin{figure}[htp]
\begin{center}
\resizebox{.2\linewidth}{!}{
\begin{tikzpicture}[->,>=stealth',shorten >=1pt,auto,node distance=2.5cm,
                    semithick]
  \tikzstyle{every state}=[draw=black,text=black]
	\tikzstyle{every edge} = [draw,thick,-]

  \node[state] 	(0)              	{1};
  \node[state]  (1) [left of=0] 	{n};
  \node[state] 	(2) [above left of=0]  {2};
  \node[state]  (3) [above of=0] 	{3};
	\node[state]  (4) [above right of=0] 	{4};
	\node[state]  (5) [right of=0] 	{5};
	\node[state]  (6) [below right of=0] 	{6};
	\node[state]  (7) [below of=0] 	{7};
	
 

  
  \path[draw=black,thick] 
  		%% drone 1
  		(0)  edge node {} (1)
			(0)  edge node {} (2)
			(0)  edge node {} (3)
			(0)  edge node {} (4)
			(0)  edge node {} (5)
			(0)  edge node {} (6)
			(0)  edge node {} (7)
  		(1)  edge[bend left = 10]  node {} (2)
  		(2)  edge[bend left = 10]  node {} (3)
  		(3)  edge[bend left = 10]  node {} (4)
			(4)  edge[bend left = 10]  node {} (5)
			(5)  edge[bend left = 10]  node {} (6)
			(6)  edge[bend left = 10]  node {} (7)
			(7)  edge[bend left = 35,draw,thick,-,dotted]  node {} (1)
  		;
\end{tikzpicture}
}
  \label{oddWheel}
\end{center}
\end{figure}
\end{proof}
%

\end{mslide}

\begin{slide}{3-Färbbarkeit und NulLA IX}

\begin{ex}[Grötzsch Graph]
\begin{figure}[htp]
\begin{center}
\resizebox{.4\linewidth}{!}{
\begin{tikzpicture}[->,>=stealth',shorten >=1pt,auto,node distance=2.5cm,
                    semithick]
  \tikzstyle{every state}=[draw=black,text=black]
	\tikzstyle{every edge} = [draw,thick,-]

  \node[state,fill=blue] 	(0)              	{1};
  \node[state, fill=yellow]  (1) [below of=0] 	{6};
  \node[state, fill=green] 	(2) [below of=1]  {11};
  \node[state, fill=red]  (3) [below right of=1] 	{7};
	\node[state,fill=green]  (4) [right of=3] 	{2};
	\node[state,fill=blue]  (5) [below right of=2] 	{8};
	\node[state,fill=blue]  (6) [below left of=2] 	{9};
	\node[state,fill=yellow]  (7) [below right of=5] 	{3};
	\node[state,fill=green]  (8) [below left of=6] 	{4};
	\node[state, fill=red]  (9) [below left of=1] 	{10};
	\node[state,fill=red]  (10) [left of=9] 	{5};

  
  \path[draw=black,thick] 
  		% drone 1
  		(0)  edge node {} (3)
			(0)  edge node {} (4)
			(0)  edge node {} (10)
			(0)  edge node {} (9)
  		(1)  edge node {} (2)
			(1)  edge node {} (4)
			(1)  edge node {} (10)
			(2)  edge node {} (3)
			(2)  edge node {} (5)
			(2)  edge node {} (6)
			(2)  edge node {} (9)
			(3)  edge node {} (7)
			(4)  edge node {} (5)
			(4)  edge node {} (7)
			(5)  edge node {} (8)
			(6)  edge node {} (10)
			(6)  edge node {} (7)
			(7)  edge node {} (8)
			(8)  edge node {} (9)
			(8)  edge node {} (10)
  		;
\end{tikzpicture}
}
\end{center}
\end{figure}
\begin{align*}
C\;:=\;\{&(1,2,3,7),(2,3,4,8),(3,4,5,9),(4,5,1,10),(1,10,11,7),\\&(2,6,11,8),(3,7,11,9),(4,8,11,10),(5,9,11,6)\}.
\end{align*}
\end{ex}

\end{slide}

\subsection{Stable Sets und NulLA}
\begin{mslide}{Stable Sets und NulLA}

\end{mslide}

\subsection{Zusammenfassung}
\begin{slide}{Zusammenfassung}

\end{slide}

