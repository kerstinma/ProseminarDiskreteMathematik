\vspace*{0.2cm}\section*{Abstract}
In dieser Proseminararbeit widmen wir uns dem Hilbert'schen Nullstellensatz, der zu den klassischen Theoremen in der Algebraischen Geometrie zählt. Im Fokus steht die Verwendung des Nullstellensatzes zum Beweisen der Unlösbarkeit von kombinatorischen Optimierungsproblemen, wie zum Beispiel der 3-Färbbarkeit eines Graphen. \\

\noindent Zu Beginn werden wir auf Grundbegriffe aus der Graphentheorie, der (Linearen) Algebra und der Komplexitätstheorie eingehen. Danach wird der Nullstellensatz und seine Anwendungen in der Diskreten Optimierung genauer behandelt. Außerdem werden Probleminstanzen aufgezeigt, welche er sehr schnell lösen kann, aber auch jene, an denen er scheitert. Außerdem werden wir einen Vergleich mit anderen Algorithmen zur Lösung der 3--Färbbarkeit aufstellen. Zum Schluss widmen wir uns noch theoretischen Resultaten und dem Beweis des Nullstellensatzes. \\

\noindent Diese Arbeit wurde von Anna Jellen und Kerstin Maier gemeinsam und zu gleichen Teilen verfasst.

%%% Local Variables: 
%%% mode: latex
%%% TeX-master: "master"
%%% End: 
